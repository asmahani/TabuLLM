\documentclass[article]{jss}

%% -- LaTeX packages and custom commands ---------------------------------------

%% recommended packages
\usepackage{orcidlink,lmodern}

%% use PythonTeX package for running Python code
\usepackage{pythontex}

%% additional package (optional)
\usepackage{framed}

%% custom commands for formatting
%\newcommand{\pkg}[1]{\texttt{#1}}
%\newcommand{\code}[1]{\texttt{#1}}
\newcommand{\fct}[1]{\texttt{#1()}}

%% -- Article metainformation (author, title, ...) -----------------------------

\author{Alireza S. Mahani~\orcidlink{0000-0002-7932-6681}\\Statman Solution Ltd.}
\Plainauthor{Alireza S. Mahani}

\title{\pkg{TabuLLM}: Feature Extraction from Tabular Data Text using Large Language Models (LLMs)}
\Plaintitle{TabuLLM: Feature Extraction from Tabular Data Text using Large Language Models (LLMs)}
\Shorttitle{Feature Extraction using LLMs}

\Abstract{
This article demonstrates how to document and showcase a Python package using 
\LaTeX\ and PythonTeX. The article includes an introduction to the package, 
its main functionalities, and some example code snippets with outputs 
generated using PythonTeX.
}

\Keywords{Python, PythonTeX, code demo, package}
\Plainkeywords{Python, PythonTeX, code demo, package}

\Address{
  Your Name\\
  Your Department\\
  Your Institution\\
  Your Address\\
  E-mail: \email{your.email@example.com}\\
  URL: \url{https://yourwebsite.com/}
}

\begin{document}

%% -- Introduction -------------------------------------------------------------
\section[Introduction to My Python Package]{Introduction to My Python Package} \label{sec:intro}

\begin{leftbar}
This article introduces the Python package \pkg{mypackage}. The package provides 
useful functions for data analysis, manipulation, and visualization. In this 
document, we demonstrate some of the core functionalities using live Python 
code examples and inline outputs generated via PythonTeX.
\end{leftbar}

The \pkg{mypackage} simplifies common tasks in Python and provides an 
intuitive interface. Below, we demonstrate the usage of its main function 
\code{add_numbers()}.

\section{Code and Demo} \label{sec:code-demo}

\subsection{Simple Function Example}

\begin{pyverbatim}
def add_numbers(a, b):
    return a + b
\end{pyverbatim}


%% -- Conclusion ---------------------------------------------------------------
\section{Conclusion} \label{sec:conclusion}

In this article, we demonstrated how to document a Python package and 
illustrate its usage with PythonTeX. The package \pkg{mypackage} offers 
functions such as \code{add_numbers()} and \code{multiply_numbers()}, making 
it easier to perform basic arithmetic operations. Future versions will include 
more advanced features for data processing.

\section*{Computational Details}

This article was compiled using PythonTeX with \LaTeX\ and Python 3.x. 
The code snippets were executed inline, and outputs were integrated into 
the document automatically.

\end{document}
